%%% -*- TeX-engine: xetex -*-
\documentclass{beamer}

% for themes, etc.
\input{preambles/packages}
\input{preambles/styles}
\usepackage{fontspec}
\usepackage{xunicode}
\usepackage[BoldFont,SlantFont]{xeCJK}
\setCJKmainfont{SimSun}
\setCJKmonofont{SimHei}

\title{
基于翻译技术的流量调度研究 \\
中期报告
}
\author{王文鑫}
\date{2016年4月15日}
% \AtBeginSection[]
% {
%   \begin{frame}<beamer> 
%     \tableofcontents[currentsection]
%   \end{frame}
% }

\begin{document}

\begin{frame}
  \titlepage
\end{frame}

\section{研究目标}

\begin{frame}
  \frametitle{研究目标}

  存在多个多种类的IPv6出口时,二级翻译器对流量的调度机制和策略
\end{frame}

\subsection{现有模型}
\begin{frame}
  \frametitle{现有模型}

  \begin{center}
    \includegraphics[width=0.8\textwidth]{figs/current-model.pdf}  
  \end{center}

  \vspace{1em}

  \begin{block}{}
    \begin{itemize}
    \item 所有流量必须先通过IVI无状态翻译
    \item DS-Lite隧道仅作为无法翻译时的后备方法
    \end{itemize}
  \end{block}
\end{frame}

\subsection{目标模型}
\begin{frame}
  \frametitle{目标模型}

  \begin{center}
    \includegraphics[width=0.8\textwidth]{figs/target-model.pdf}  
  \end{center}

  \begin{block}{}
    \begin{itemize}
    \item 按照流信息、链路状态等参数向不同出口分发流量
    \item 无状态翻译仍可能失败
    \item 不能只使用目标地址的路由进行调度
    \end{itemize}
  \end{block}
\end{frame}

\section{翻译器设计}
\begin{frame}
  \frametitle{翻译器设计}

  \begin{center}
    \includegraphics[width=0.5\textwidth]{figs/xlat-model.pdf}  
  \end{center}

  \begin{block}{}
    \begin{itemize}
    \item 按照流信息、链路状态等参数向不同出口分发流量
    \end{itemize}
  \end{block}
\end{frame}

\section{完成工作}
\begin{frame}
  \frametitle{完成工作}

\end{frame}

\section{后续工作}
\begin{frame}
  \frametitle{后续工作}

\end{frame}

\section{Q\&A}

\begin{frame}
  \frametitle{Q\&A}
  \begin{center}
  {\LARGE 谢谢!}
  \vspace{3em}
  {\LARGE 请老师们提问和指导!}
  \end{center}
\end{frame}
\end{document}